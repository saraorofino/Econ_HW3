\documentclass[]{article}
\usepackage{lmodern}
\usepackage{amssymb,amsmath}
\usepackage{ifxetex,ifluatex}
\usepackage{fixltx2e} % provides \textsubscript
\ifnum 0\ifxetex 1\fi\ifluatex 1\fi=0 % if pdftex
  \usepackage[T1]{fontenc}
  \usepackage[utf8]{inputenc}
\else % if luatex or xelatex
  \ifxetex
    \usepackage{mathspec}
  \else
    \usepackage{fontspec}
  \fi
  \defaultfontfeatures{Ligatures=TeX,Scale=MatchLowercase}
\fi
% use upquote if available, for straight quotes in verbatim environments
\IfFileExists{upquote.sty}{\usepackage{upquote}}{}
% use microtype if available
\IfFileExists{microtype.sty}{%
\usepackage{microtype}
\UseMicrotypeSet[protrusion]{basicmath} % disable protrusion for tt fonts
}{}
\usepackage[margin=1in]{geometry}
\usepackage{hyperref}
\hypersetup{unicode=true,
            pdftitle={ESM 204 Assignment 3},
            pdfauthor={Jeremy Knox, Sara Orofino},
            pdfborder={0 0 0},
            breaklinks=true}
\urlstyle{same}  % don't use monospace font for urls
\usepackage{graphicx,grffile}
\makeatletter
\def\maxwidth{\ifdim\Gin@nat@width>\linewidth\linewidth\else\Gin@nat@width\fi}
\def\maxheight{\ifdim\Gin@nat@height>\textheight\textheight\else\Gin@nat@height\fi}
\makeatother
% Scale images if necessary, so that they will not overflow the page
% margins by default, and it is still possible to overwrite the defaults
% using explicit options in \includegraphics[width, height, ...]{}
\setkeys{Gin}{width=\maxwidth,height=\maxheight,keepaspectratio}
\IfFileExists{parskip.sty}{%
\usepackage{parskip}
}{% else
\setlength{\parindent}{0pt}
\setlength{\parskip}{6pt plus 2pt minus 1pt}
}
\setlength{\emergencystretch}{3em}  % prevent overfull lines
\providecommand{\tightlist}{%
  \setlength{\itemsep}{0pt}\setlength{\parskip}{0pt}}
\setcounter{secnumdepth}{0}
% Redefines (sub)paragraphs to behave more like sections
\ifx\paragraph\undefined\else
\let\oldparagraph\paragraph
\renewcommand{\paragraph}[1]{\oldparagraph{#1}\mbox{}}
\fi
\ifx\subparagraph\undefined\else
\let\oldsubparagraph\subparagraph
\renewcommand{\subparagraph}[1]{\oldsubparagraph{#1}\mbox{}}
\fi

%%% Use protect on footnotes to avoid problems with footnotes in titles
\let\rmarkdownfootnote\footnote%
\def\footnote{\protect\rmarkdownfootnote}

%%% Change title format to be more compact
\usepackage{titling}

% Create subtitle command for use in maketitle
\newcommand{\subtitle}[1]{
  \posttitle{
    \begin{center}\large#1\end{center}
    }
}

\setlength{\droptitle}{-2em}

  \title{ESM 204 Assignment 3}
    \pretitle{\vspace{\droptitle}\centering\huge}
  \posttitle{\par}
    \author{Jeremy Knox, Sara Orofino}
    \preauthor{\centering\large\emph}
  \postauthor{\par}
      \predate{\centering\large\emph}
  \postdate{\par}
    \date{5/15/2019}


\begin{document}
\maketitle

\subsubsection{1. Linear Probability
Model}\label{linear-probability-model}

Create a linear probility model that predicts a respondent's probability
of voting ``yes'' on the ballot based on their age, income, NEP score,
the program's risk reduction, and cost of the program to that
respondent.

\textbf{Regression Model:}

\(Probability(Voting~Yes) = 0.1197 + 0.0204(Age~to~30) - 0.0201(Age~to~40) + 0.01(Age~to~50) - 0.0162(Age~to~60) + 0.0088(Income~One~Percent) + 0.0027(Income~Poor) + 0.0075(Income~Rich) + 0.0468(Income~Very~Rich) + 0.0159(NEP) - 0.0011(Bid) + 7\times 10^{-4}(Risk~Reduction)\)

\textbf{Coefficient Interpretation:}\\
\emph{Age: Reference Level Over 65}\\
- to 30:\\
- to 40:\\
- to 50:\\
- to 60:\\
\emph{Income: Reference Level Middle}\\
- One Percent:\\
- Poor:\\
- Rich:\\
- Very Rich:\\
\emph{NEP:}\\
\emph{Bid:}\\
\emph{Risk:}

\subsubsection{2. Value of Prevented Whale
Deaths}\label{value-of-prevented-whale-deaths}

Reducing the risk of whale strikes by 20\% saves five whales every year.
Based on this, the vessel speed reduction by 4\% saves a single whale
every year. To find the value of each individual whale saved find the
willingess to pay for vessel speed reduction programs of 0\% and compare
to the willingness to pay for vessel speed reduction of 4\%.

\textbf{Risk Reduction 0\%}\\
Assume the probability of voting yes is the average of the votes
(\(p = 0.714\)), assume an age to 30, income rich, and the average NEP
(38.366), solve for the willingness to pay for the program using:

\(0.714 = 0.1197 + 0.0204(Age~to~30) + 0.0075(Income~Rich) + 0.0159(NEP) - 0.0011(Bid) + 7\times 10^{-4}(0)\)

\(Willingness~to~Pay = 39.5002\)

\textbf{Risk Reduction 4\%}\\
Again, assume the probability of voting yes is the average of the votes
(\(p = 0.714\)), assume an age to 30, income rich, and the average NEP
(38.366), solve for the willingness to pay for the program using:

\(0.714 = 0.1197 + 0.0204(Age~to~30) + 0.0075(Income~Rich) + 0.0159(NEP) - 0.0011(Bid) + 7\times 10^{-4}(4)\)

\(Willingness~to~Pay = 42.2836\)

The value of a single whale is the difference between the willingness to
pay for a vessel speed reduction program at 4\% and at 0\%.

\(Individual~Whale~Value = 2.7834\)

\subsubsection{3. Estimated Willingness to Pay for a Vessel Speed
Reduction
Program}\label{estimated-willingness-to-pay-for-a-vessel-speed-reduction-program}

\paragraph{a. Choose three participants at
random}\label{a.-choose-three-participants-at-random}

Using a random number generator select three participants:\\
- 38 NEP:32 Income:Rich Age:to30\\
- 44 NEP:51 Income:Poor Age:to40\\
- 102 NEP:51 Income:Middle Age:to60

\subsubsection{b. Predict willingness to pay for 60\% VSR
program}\label{b.-predict-willingness-to-pay-for-60-vsr-program}

Assume the probability of voting yes the average of all the yes votes
(\(p = 0.714\)), calculate the willingness to pay using the following
equations:

Individual 38:\\
\(0.714 = 0.1197 + 0.0204(Age~to~30) + 0.0075(Income~Rich) + 0.0159(NEP) - 0.0011(Bid) + 7\times 10^{-4}(60)\)

Individual 44:\\
\(0.714 = 0.1197 - 0.0201(Age~to~40) + 0.0027(Income~Poor) + 0.0159(NEP) - 0.0011(Bid) + 7\times 10^{-4}(60)\)

Individual 122:\\
\(0.714 = 0.1197- 0.0162(Age~to~60) + 0.0159(NEP) - 0.0011(Bid)+ 7\times 10^{-4}(60)\)

Individual 38:\\
\(Willingness~to~Pay = 13.1408\)

Individual 44:\\
\(Willingness~to~Pay = 263.8417\)

Individual 122:\\
\(Willingness~to~Pay = 257.6434\)

\subsubsection{4. Santa Barbara Estimated Willingness to Pay for VSR
Program}\label{santa-barbara-estimated-willingness-to-pay-for-vsr-program}

Again assume the probability of voting yes the average of voting yes
(\(p = 0.714\)). Use the average income bracket (middle), the average
age bracket (to 50), and average NEP (38.366) to calculate willingness
to pay using:

\(0.714 = 0.1197 + 0.01(Age~to~50) + 0.0159(NEP) - 0.0011(Bid) + 7\times 10^{-4}(60)\)

\(Average~Santa~Barbara~County~Household~Willingness~to~Pay = `round(abs(r bid_sb), digits=4)`\)

\subsubsection{5. Total benefits to Santa Barbara residents (population
=
150,000).}\label{total-benefits-to-santa-barbara-residents-population-150000.}

Risk = 60\%\\
Cost = \$7 million

Calculate Total Benefits using the following equation:\\
Total Benefits = \(64.4760135per household * 150,000 households\)

Total Benefits = \(9.671402\times 10^{6}\)

\subsubsection{6. Do Benefits Outweight
Costs}\label{do-benefits-outweight-costs}

Based only on these benefits the benefits do outweight the costs.

Cost = \$7 million\\
Benefits = 9.671402\times 10\^{}\{6\}

\(Surplus = 9.671402\times 10^{6} - 7000000 = 2.671402\times 10^{6}\)

\subsubsection{7. Price of Carbon
Credits}\label{price-of-carbon-credits}

Assume that for any ship transiting the Santa Barbara Channel the speed
reduction tha results in a 60\% risk reduction costs the shipper \$1000
but will result in 20 fewer tons of carbon dioxide per transit.

Shipping companies will adopt the VSR for self interested reasons if the
value of the carbon credits is equal to the cost.

\(20*Z = 1000\)\\
\(Z = 50/ton\)

\subsubsection{8. Whales Saved by Carbon
Credits}\label{whales-saved-by-carbon-credits}

Assume the value of the carbon credit is \$50/ton and all ships reduce
speed to achieve the 60\% risk reduction.

Number of Whales Saved:\\
20\% Reduction = 5 Whales saved\\
\(Whales Saved = 5*3 = 15\)

The social value of the whale reduction program is the total benefit of
that program minus the cost. Since the carbon credits would avoid the
\$7 million cost of the program, the social value is just the total
benefit of the program.

\(Social value = 9.671402\times 10^{6}\)


\end{document}
