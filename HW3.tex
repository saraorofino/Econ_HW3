\documentclass[10pt,]{article}
\usepackage{lmodern}
\usepackage{amssymb,amsmath}
\usepackage{ifxetex,ifluatex}
\usepackage{fixltx2e} % provides \textsubscript
\ifnum 0\ifxetex 1\fi\ifluatex 1\fi=0 % if pdftex
  \usepackage[T1]{fontenc}
  \usepackage[utf8]{inputenc}
\else % if luatex or xelatex
  \ifxetex
    \usepackage{mathspec}
  \else
    \usepackage{fontspec}
  \fi
  \defaultfontfeatures{Ligatures=TeX,Scale=MatchLowercase}
\fi
% use upquote if available, for straight quotes in verbatim environments
\IfFileExists{upquote.sty}{\usepackage{upquote}}{}
% use microtype if available
\IfFileExists{microtype.sty}{%
\usepackage{microtype}
\UseMicrotypeSet[protrusion]{basicmath} % disable protrusion for tt fonts
}{}
\usepackage[margin=1in]{geometry}
\usepackage{hyperref}
\PassOptionsToPackage{usenames,dvipsnames}{color} % color is loaded by hyperref
\hypersetup{unicode=true,
            colorlinks=true,
            linkcolor=Maroon,
            citecolor=Blue,
            urlcolor=blue,
            breaklinks=true}
\urlstyle{same}  % don't use monospace font for urls
\usepackage{graphicx,grffile}
\makeatletter
\def\maxwidth{\ifdim\Gin@nat@width>\linewidth\linewidth\else\Gin@nat@width\fi}
\def\maxheight{\ifdim\Gin@nat@height>\textheight\textheight\else\Gin@nat@height\fi}
\makeatother
% Scale images if necessary, so that they will not overflow the page
% margins by default, and it is still possible to overwrite the defaults
% using explicit options in \includegraphics[width, height, ...]{}
\setkeys{Gin}{width=\maxwidth,height=\maxheight,keepaspectratio}
\IfFileExists{parskip.sty}{%
\usepackage{parskip}
}{% else
\setlength{\parindent}{0pt}
\setlength{\parskip}{6pt plus 2pt minus 1pt}
}
\setlength{\emergencystretch}{3em}  % prevent overfull lines
\providecommand{\tightlist}{%
  \setlength{\itemsep}{0pt}\setlength{\parskip}{0pt}}
\setcounter{secnumdepth}{0}
% Redefines (sub)paragraphs to behave more like sections
\ifx\paragraph\undefined\else
\let\oldparagraph\paragraph
\renewcommand{\paragraph}[1]{\oldparagraph{#1}\mbox{}}
\fi
\ifx\subparagraph\undefined\else
\let\oldsubparagraph\subparagraph
\renewcommand{\subparagraph}[1]{\oldsubparagraph{#1}\mbox{}}
\fi

%%% Use protect on footnotes to avoid problems with footnotes in titles
\let\rmarkdownfootnote\footnote%
\def\footnote{\protect\rmarkdownfootnote}

%%% Change title format to be more compact
\usepackage{titling}

% Create subtitle command for use in maketitle
\newcommand{\subtitle}[1]{
  \posttitle{
    \begin{center}\large#1\end{center}
    }
}

\setlength{\droptitle}{-2em}

  \title{}
    \pretitle{\vspace{\droptitle}}
  \posttitle{}
    \author{}
    \preauthor{}\postauthor{}
    \date{}
    \predate{}\postdate{}
  
\usepackage{crimson}
\usepackage[T1]{fontenc}
\usepackage[margin=1in]{geometry}
\renewcommand{\baselinestretch}{1}
\usepackage{calc}
\usepackage{enumitem}
\usepackage{changepage}

\begin{document}

\vspace{-85pt} \text{\bfseries\LARGE{MEMORANDUM}} \newline
\vspace{-18pt} \noindent\makebox[\linewidth]{\rule{\textwidth}{0.4pt}}
\vspace{-18pt} \setlist[2]{noitemsep}

\begin{description}[leftmargin=!,itemsep=-1ex,labelwidth=\widthof{Subject:  }]
  \item[To:] Santa Barbara Government 
  \item[From:] Sara Orofino and Jeremy Knox, Master of Environmental Science and Management Students
  \item[Date:] May 22, 2019
  \item[Subject:] \bfseries{Economics of a Vessel Speed Reduction Program in Santa Barbara Channel}
\end{description}

\vspace{-10pt} \noindent\makebox[\linewidth]{\rule{\textwidth}{0.4pt}}
\vspace{-25pt}

\subsection{BACKGROUND}\label{background}

\vspace{-10pt}

The coast of California is home to a diverse marine ecosystem. Among the
most important components are whales. Every year, over 20 whales are
struck by container ships along the West Coast of the United
States.\footnote{Redfern et al., 2013 cited from WhaleStrikes Group
  Porject} The Santa Barbara Channel, off the coast of Calfiornia
stretching from Point Conception to Point Mugu, is frequently visited by
three endangered whale species: blue, fin and humpback. This is due to
the unique location of the SB Channel where cold water from the north
combines with warmer water from the south. Unfortunately, this channel
is also a key shipping area for container ships. These ships transport
90\% of the world's goods and travel through SB Channel more than 2,700
times every year\footnote{Hakim, 2014 cited from WhaleStrikes Group
  Porject}. One proposed soltuion to reduce the frequency of whale
strikes is a vessel speed reduction (VSR) program. Reducing the speed of
ships through this channel would result in a 60\% decrease in risk of
whale strikes. The following analysis demonstrates that VSR program
benefits outweigh costs by a couple orders of magnitude and concludes
that the prgram should be adopted.

\vspace{-15pt} \#\# METHODS \vspace{-10pt}

\textbf{Data}: A contingent valuation survey to 500 Santa Barbara County
residents was given to determine the local demand for a VSR program.
Variables measured: (1) \emph{risk}: level of risk reduction, (2)
\emph{bid}: annual payment for the household, (3) \emph{vote}: 1 is yes,
0 is no, (4) \emph{NEP} : measure of environmental concern, (5)
\emph{age}: categorical, (6) \emph{income}: categorical.\footnote{WhaleStrikes
  Group Porject} \textbf{Model}: Linear regression model of \emph{vote}
on \emph{age}, \emph{income}, \emph{NEP}, \emph{bid} and
\emph{risk}.\footnote{See Technical Appendix: (1) Linear model is used,
  instead of logistic, because the ease of interpretation. Significance
  of variables was not analyzed and averages/medians were used for
  calculations.}

\vspace{-15pt} \#\# ANALYSIS \vspace{-10pt}

To analyze Santa Barbara respondents probability of voting yes on a VSR
program we use a linear regression model. Given that reducing the risk
of whales strikes by 20\% saves five whales, our model suggests the
willingness to pay for saving one whale is \$2.78, on average ceteris
paribus.\footnote{See Technical Appendix: (2)} Furthermore, implementing
a 60\% reduction VSR program would result in an average willingness to
pay of \$71.48.\footnote{See Technical Appendix: (4)} Reducing the risk
by this much would save 15 whales. Santa Barbara residents will recieve
these benefits but the cost will be spread across the country (from a
price increase on goods transported by these ships).The cost to
implement a 60\% VSR is \$7 million. Santa Barbara residents would
recieve a benefit of over \$10 billion giving a surplus of over \$3.7
million.\footnote{See Technical Appendix: (5) \& (6)}

Ships that reduce their speed not only reduce the risk of colliding with
whales but also emit less carbon. An alternative to the price increase
on products aboard these ships is a carbon credit program. There are
carbon trading markets that could potentially offset the cost of
implmenting a 60\% VSR program. Our analysis shows that a carbon market
that prices carbon at \$50 per ton would offset the cost of reducing
speed thereby incentivising ships to reduce their speed.\footnote{See
  Technical Appendix: (7)} In this scenario, the benefit of 15 whales
being saved to Santa Barbara Residents would be over \$10 billion while
the cost would go to all tax payers and stakeholders who fund the carbon
offeset program.

\vspace{-15pt} \#\# CONCLUSION \vspace{-10pt}

Reducing the speed of ships through the Santa Barbara Channel would
result in a 60\% decrease in risk of whale strikes, ultimatley saving 15
whales. The cost would fall on consumers of the products among these
ships or citizens whose tax dollars go to a carbon credit program. Given
that benefits far outweigh cost, Santa Barbara Government should
implment the VSR program.

\newpage

\subsection{Technical Appendix}\label{technical-appendix}

\vspace{-10pt}

\subsubsection{1. Linear Probability
Model}\label{linear-probability-model}

Create a linear probility model that predicts a respondent's probability
of voting ``yes'' on the ballot based on their age, income, NEP score,
the program's risk reduction, and cost of the program to that
respondent.

\begin{table}[!htbp] \centering 
  \caption{Linear Regression Results} 
  \label{} 
\begin{tabular}{@{\extracolsep{5pt}}lc} 
\\[-1.8ex]\hline 
\hline \\[-1.8ex] 
 & \multicolumn{1}{c}{\textit{Dependent variable:}} \\ 
\cline{2-2} 
\\[-1.8ex] & Vote Yes \\ 
\hline \\[-1.8ex] 
 agetofifty & 0.0100 \\ 
  & (0.0633) \\ 
  & \\ 
 agetoforty & $-$0.0201 \\ 
  & (0.0624) \\ 
  & \\ 
 agetosixty & $-$0.0162 \\ 
  & (0.0596) \\ 
  & \\ 
 agetothirty & 0.0204 \\ 
  & (0.0578) \\ 
  & \\ 
 incomeone\_percent & 0.0088 \\ 
  & (0.0599) \\ 
  & \\ 
 incomepoor & 0.0027 \\ 
  & (0.0650) \\ 
  & \\ 
 incomerich & 0.0075 \\ 
  & (0.0682) \\ 
  & \\ 
 incomevery\_rich & 0.0468 \\ 
  & (0.0675) \\ 
  & \\ 
 NEP & 0.0159$^{***}$ \\ 
  & (0.0021) \\ 
  & \\ 
 bid & $-$0.0011 \\ 
  & (0.0007) \\ 
  & \\ 
 risk & 0.0007 \\ 
  & (0.0008) \\ 
  & \\ 
 Constant & 0.1197 \\ 
  & (0.1199) \\ 
  & \\ 
\hline \\[-1.8ex] 
Observations & 500 \\ 
R$^{2}$ & 0.1201 \\ 
Adjusted R$^{2}$ & 0.1003 \\ 
Residual Std. Error & 0.4291 (df = 488) \\ 
F Statistic & 6.0554$^{***}$ (df = 11; 488) \\ 
\hline 
\hline \\[-1.8ex] 
\textit{Note:}  & \multicolumn{1}{r}{$^{*}$p$<$0.1; $^{**}$p$<$0.05; $^{***}$p$<$0.01} \\ 
\end{tabular} 
\end{table}

{[}1{]} ``''\\
{[}2{]} ``\textbackslash{}begin\{table\}{[}!htbp{]}
\textbackslash{}centering''\\
{[}3{]} " \textbackslash{}caption\{Linear Regression Results\} "\\
{[}4{]} " \textbackslash{}label\{\} "\\
{[}5{]}
``\textbackslash{}begin\{tabular\}\{@\{\textbackslash{}extracolsep\{5pt\}\}lc\}''\\
{[}6{]}
``\textbackslash{}\textbackslash{}{[}-1.8ex{]}\textbackslash{}hline''\\
{[}7{]} ``\textbackslash{}hline
\textbackslash{}\textbackslash{}{[}-1.8ex{]}''\\
{[}8{]} " \&
\textbackslash{}multicolumn\{1\}\{c\}\{\textbackslash{}textit\{Dependent
variable:\}\} \textbackslash{}\textbackslash{} "\\
{[}9{]} ``\textbackslash{}cline\{2-2\}''\\
{[}10{]} ``\textbackslash{}\textbackslash{}{[}-1.8ex{]} \& Vote Yes
\textbackslash{}\textbackslash{}''\\
{[}11{]} ``\textbackslash{}hline
\textbackslash{}\textbackslash{}{[}-1.8ex{]}''\\
{[}12{]} " agetofifty \& 0.0100 \textbackslash{}\textbackslash{} "\\
{[}13{]} " \& (0.0633) \textbackslash{}\textbackslash{} "\\
{[}14{]} " \& \textbackslash{}\textbackslash{} "\\
{[}15{]} " agetoforty \& \$-\$0.0201 \textbackslash{}\textbackslash{}
"\\
{[}16{]} " \& (0.0624) \textbackslash{}\textbackslash{} "\\
{[}17{]} " \& \textbackslash{}\textbackslash{} "\\
{[}18{]} " agetosixty \&
\$-\(0.0162 \\\\ " [19] " & (0.0596) \\\\ " [20] " & \\\\ " [21] " agetothirty & 0.0204 \\\\ " [22] " & (0.0578) \\\\ " [23] " & \\\\ " [24] " incomeone\\_percent & 0.0088 \\\\ " [25] " & (0.0599) \\\\ " [26] " & \\\\ " [27] " incomepoor & 0.0027 \\\\ " [28] " & (0.0650) \\\\ " [29] " & \\\\ " [30] " incomerich & 0.0075 \\\\ " [31] " & (0.0682) \\\\ " [32] " & \\\\ " [33] " incomevery\\_rich & 0.0468 \\\\ " [34] " & (0.0675) \\\\ " [35] " & \\\\ " [36] " NEP & 0.0159\)\^{}\{***\}\$
\textbackslash{}\textbackslash{} "\\
{[}37{]} " \& (0.0021) \textbackslash{}\textbackslash{} "\\
{[}38{]} " \& \textbackslash{}\textbackslash{} "\\
{[}39{]} " bid \&
\$-\(0.0011 \\\\ " [40] " & (0.0007) \\\\ " [41] " & \\\\ " [42] " risk & 0.0007 \\\\ " [43] " & (0.0008) \\\\ " [44] " & \\\\ " [45] " Constant & 0.1197 \\\\ " [46] " & (0.1199) \\\\ " [47] " & \\\\ " [48] "\\hline \\\\[-1.8ex] " [49] "Observations & 500 \\\\ " [50] "R\)\^{}\{2\}\$
\& 0.1201 \textbackslash{}\textbackslash{} "\\
{[}51{]} ``Adjusted R\(^{2}\) \& 0.1003
\textbackslash{}\textbackslash{}''\\
{[}52{]} ``Residual Std. Error \& 0.4291 (df = 488)
\textbackslash{}\textbackslash{}''\\
{[}53{]} ``F Statistic \& 6.0554\(^{***}\) (df = 11; 488)
\textbackslash{}\textbackslash{}''\\
{[}54{]} ``\textbackslash{}hline''\\
{[}55{]} ``\textbackslash{}hline
\textbackslash{}\textbackslash{}{[}-1.8ex{]}''\\
{[}56{]} ``\textbackslash{}textit\{Note:\} \&
\textbackslash{}multicolumn\{1\}\{r\}\{\(^{*}\)p\$\textless{}\$0.1;
\(^{**}\)p\$\textless{}\$0.05; \(^{***}\)p\$\textless{}\$0.01\}
\textbackslash{}\textbackslash{}'' {[}57{]}
``\textbackslash{}end\{tabular\}''\\
{[}58{]} ``\textbackslash{}end\{table\}''

\textbf{Regression Model:}

\(Probability(Voting~Yes) = 0.1197 + 0.0204(Age~to~30) - 0.0201(Age~to~40) + 0.01(Age~to~50) - 0.0162(Age~to~60) + 0.0088(Income~One~Percent) + 0.0027(Income~Poor) + 0.0075(Income~Rich) + 0.0468(Income~Very~Rich) + 0.0159(NEP) - 0.0011(Bid) + 7\times 10^{-4}(Risk~Reduction)\)

\textbf{Coefficient Interpretation:}\\
\emph{Age: Reference Level Over 65}\\
- to 30: All else being equal, a person in the age bracket of to 30
would be expected to have a probability of voting yes on a vessel speed
reduction program that is, on average, \(0.0204\) higher than a person
in the age bracket over 65.\\
- to 40: All else being equal, a person in the age bracket of to 40
would be expected to have a probability of voting yes on a vessel speed
reduction program that is, on average, \(0.0201\) lower than a person in
the age bracket over 65.\\
- to 50: All else being equal, a person in the age bracket of to 50
would be expected to have a probability of voting yes on a vessel speed
reduction program that is, on average, \(0.01\) higher than a person in
the age bracket over 65.\\
- to 60: All else being equal, a person in the age bracket of to 60
would be expected to have a probability of voting yes on a vessel speed
reduction program that is, on average, \(0.0162\) lower than a person in
the age bracket over 65.

\emph{Income: Reference Level Middle}\\
- Poor: All else being equal, a person with an income level of one
percent would be expected to have a probability of voting yes on a
vessel speed reduction program that is, on average, \(0.0027\) higher
than a person with a medium income level.\\
- Rich: All else being equal, a person with an income level of one
percent would be expected to have a probability of voting yes on a
vessel speed reduction program that is, on average, \(0.0075\) higher
than a person with a medium income level.\\
- Very Rich: All else being equal, a person with an income level of one
percent would be expected to have a probability of voting yes on a
vessel speed reduction program that is, on average, \(0.0468\) higher
than a person with a medium income level.\\
- One Percent: All else being equal, a person with an income level of
one percent would be expected to have a probability of voting yes on a
vessel speed reduction program that is, on average, \(0.0088\) higher
than a person with a medium income level.

\emph{NEP:} For every 1 unit increase in environmental concern, we would
expect the probability of voting yes on a vessel speed reduction program
to increase by \(0.0159\), if all other variables are equal.

\emph{Bid:} For every 1 dollar increase in annual household payment, we
expect the probability of voting yes on a vessel speed reduction program
to decrease by \(0.0011\), if all other variables are equal.

\emph{Risk:} For every 1 unit increase in risk reduction, we would
expect the probability of voting yes on a vessel speed reduction program
to increase by \(7\times 10^{-4}\), if all other variables are equal.

\subsubsection{2. Value of Prevented Whale
Deaths}\label{value-of-prevented-whale-deaths}

Reducing the risk of whale strikes by 20\% saves five whales every year.
Based on this, the vessel speed reduction by 4\% saves a single whale
every year. To find the value of each individual whale saved find the
willingess to pay for vessel speed reduction programs of 0\% and compare
to the willingness to pay for vessel speed reduction of 4\%.

\textbf{Risk Reduction 0\%}\\
Assume the probability of voting yes is 0.5, assume an age to 30, income
rich, and the average NEP (38.366), solve for the willingness to pay for
the program using:

\(0.5 = 0.1197 + 0.0204(Age~to~30) + 0.0075(Income~Rich) + 0.0159(NEP) - 0.0011(Bid) + 7\times 10^{-4}(0)\)

\(Willingness~to~Pay = 239.52\)

\textbf{Risk Reduction 4\%}\\
Again, assume the probability of voting yes is 0.5, assume an age to 30,
income rich, and the average NEP (38.366), solve for the willingness to
pay for the program using:

\(0.5 = 0.1197 + 0.0204(Age~to~30) + 0.0075(Income~Rich) + 0.0159(NEP) - 0.0011(Bid) + 7\times 10^{-4}(4)\)

\(Willingness~to~Pay = 242.3\)

The value of a single whale is the difference between the willingness to
pay for a vessel speed reduction program at 4\% and at 0\%.

\(Individual~Whale~Value = 2.78\)

\subsubsection{3. Estimated Willingness to Pay for a Vessel Speed
Reduction
Program}\label{estimated-willingness-to-pay-for-a-vessel-speed-reduction-program}

\paragraph{a. Choose three participants at
random}\label{a.-choose-three-participants-at-random}

Using a random number generator select three participants:\\
- 38 NEP:32 Income:Rich Age:to30\\
- 44 NEP:51 Income:Poor Age:to40\\
- 102 NEP:51 Income:Middle Age:to60

\paragraph{b. Predict willingness to pay for 60\% VSR
program}\label{b.-predict-willingness-to-pay-for-60-vsr-program}

Assume the probability of voting yes the average of all the yes votes
(\(p = 0.714\)), calculate the willingness to pay using the following
equations:

Individual 38:\\
\(0.5 = 0.1197 + 0.0204(Age~to~30) + 0.0075(Income~Rich) + 0.0159(NEP) - 0.0011(Bid) + 7\times 10^{-4}(60)\)

Individual 44:\\
\(0.5 = 0.1197 - 0.0201(Age~to~40) + 0.0027(Income~Poor) + 0.0159(NEP) - 0.0011(Bid) + 7\times 10^{-4}(60)\)

Individual 122:\\
\(0.5 = 0.1197- 0.0162(Age~to~60) + 0.0159(NEP) - 0.0011(Bid)+ 7\times 10^{-4}(60)\)

Individual 38:\\
\(Willingness~to~Pay = 186.88\)

Individual 44:\\
\(Willingness~to~Pay = 463.86\)

Individual 122:\\
\(Willingness~to~Pay = 457.66\)

\subsubsection{4. Santa Barbara Estimated Willingness to Pay for VSR
Program}\label{santa-barbara-estimated-willingness-to-pay-for-vsr-program}

Again assume the probability of voting yes is 0.5. Use the median income
bracket (rich), the median age bracket (to 50), and average NEP (38.366)
to calculate willingness to pay using:

\(0.5 = 0.1197 + 0.0075(Income~Rich) + 0.01(Age~to~50) + 0.0159(NEP) - 0.0011(Bid) + 7\times 10^{-4}(60)\)

\(Average~Santa~Barbara~County~Household~Willingness~to~Pay = 271.5\)

\subsubsection{5. Total benefits to Santa Barbara residents (population
=
150,000).}\label{total-benefits-to-santa-barbara-residents-population-150000.}

Risk = 60\%\\
Cost = 7 million

Calculate Total Benefits using the following equation:\\
Total Benefits = \(271.5~per~household * 150,000~households\)

Total Benefits = \(4.072431\times 10^{7}\)

\subsubsection{6. Do Benefits Outweight
Costs}\label{do-benefits-outweight-costs}

Yes, based only on these benefits the benefits do outweight the costs.

Cost = 7 million\\
Benefits = \(4.072431\times 10^{7}\)

\(Surplus = 4.072431\times 10^{7} - 7000000 = 3.372431\times 10^{7}\)

\subsubsection{7. Price of Carbon
Credits}\label{price-of-carbon-credits}

Assume that for any ship transiting the Santa Barbara Channel the speed
reduction tha results in a 60\% risk reduction costs the shipper 1000
but will result in 20 fewer tons of carbon dioxide per transit.

Shipping companies will adopt the VSR for self interested reasons if the
value of the carbon credits is equal to the cost.

\(20*Z = 1000\)\\
\(Z = 50/ton\)

\subsubsection{8. Whales Saved by Carbon
Credits}\label{whales-saved-by-carbon-credits}

Assume the value of the carbon credit is 50/ton and all ships reduce
speed to achieve the 60\% risk reduction.

Number of Whales Saved:\\
20\% Reduction = 5 Whales saved\\
\(Whales~Saved = 5*3 = 15\)

The social value of the whale reduction program is the total benefit of
that program minus the cost. Since the carbon credits would avoid the 7
million cost of the program, the social value is just the total benefit
of the program.

\(Social~Value = 4.072431\times 10^{7}\)


\end{document}
